\documentclass[runningheads]{llncs}
\usepackage[T5]{fontenc}
% \usepackage{graphicx}
\usepackage{hyperref}
\usepackage{xcolor}
\renewcommand\UrlFont{\color{blue}\rmfamily}

\newcommand{\vangiang}[1]{\textcolor{magenta}{#1}}
\newcommand{\sylvain}[1]{\textcolor{teal}{#1}}

\begin{document}
%
\title{Minimal trap-spaces of Logical models are maximal siphons of their Petri-net encoding}
\titlerunning{Minimal trap-spaces as maximal siphons}
\author{Van-Giang Trịnh\inst{1}\orcidID{0000-0001-6581-998X} \and \\
  Sylvain Soliman\inst{2}\orcidID{0000-0001-5525-7418}}
% \authorrunning{F. Author et al.}

\institute{
  Aix-Marseille Université, Marseille, France\\
  \email{Trinh.Van-Giang@lis-lab.fr}
  \and
  Lifeware team, Inria Saclay center, Palaiseau, France\\
  \email{Sylvain.Soliman@inria.fr}%\\
  %\url{http://lifeware.inria.fr/~soliman/}
}

\maketitle

\begin{abstract}
  % The abstract should briefly summarize the contents of the paper in
  % 150--250 words.

  Discrete modelling has proven over the years that it can bring powerful analyses and corresponding insight to the many cases where precise biological data is not sufficiently available to build a detailed quantitative model.
  This is even more true for very big models where such data is frequently missing and led to a constant increase in size of logical models \emph{à la} Thomas.
  The analysis of such models is mostly based on attractor computation, since those correspond roughly to \emph{phenotypes}, and the recent use of trap-spaces made a real breakthrough in that domain allowing to consider medium-sized models that used to be out of reach.
  However, with the continuing increase in model-size, the state of the art computation of minimal trap-spaces based on prime-implicants shows its limits as there can be very many such implicants.
  In this article we present an alternative method to compute minimal trap-spaces, and hence complex attractors, that relies on a completely different technique, namely the enumeration of maximal siphons in the Petri-net encoding of the original Logical model.
  We then demonstrate its efficiency and compare it to implicant-based methods on a few relevant big Boolean models.

\keywords{Logical models \and Boolean models \and Trap-spaces \and Attractor computation \and Petri-nets \and Siphons}
\end{abstract}

\section{Introduction}

\sylvain{We could go for a completely different title, e.g. focused on performance much more, like ``Scaling-up attractor computation for Logical models''}

\cite{glass1973logical,thomas1973boolean,thomas1990biological}

Discrete modelling has proven over the years that it can bring powerful analyses and corresponding insight to the many cases where precise biological data is not sufficiently available to build a detailed quantitative model.
This is even more true for very big models where such data is frequently missing and led to a constant increase in size of logical models \emph{à la} Thomas.
The analysis of such models is mostly based on attractor computation, since those correspond roughly to \emph{phenotypes}, and the recent use of trap-spaces made a real breakthrough in that domain allowing to consider medium-sized models that used to be out of reach.
However, with the continuing increase in model-size, the state of the art computation of minimal trap-spaces based on prime-implicants shows its limits as there can be very many such implicants.
In this article we present an alternative method to compute minimal trap-spaces, and hence complex attractors, that relies on a completely different technique, namely the enumeration of maximal siphons in the Petri-net encoding of the original Logical model.
We then demonstrate its efficiency and compare it to implicant-based methods on a few relevant big Boolean models.

\section{Minimal trap-spaces as maximal siphons}
\subsection{Preliminaries}
\subsubsection{Traps-spaces}

Purely boolean, asynchronous.

\cite{klarner2015computing,klarner2017pyboolnet,cifuentes2020control}

\subsubsection{Petri-net encoding of Logical models}
\label{sec:encoding}

The link between Logical models \emph{à la} Thomas and Petri-nets was originally established in~\cite{chaouiya2004qualitative} in order to make available formal methods like model-checking for the analysis of such systems.
The basic encoding only holds for purely Boolean models but was later extended to multi-valued models in two ways, either in \cite{chaouiya2011petri} with non 1-safe Petri-nets or more recently in~\cite{chatain2014characterization} with 1-safe nets but many more places.

Since our study is focused on Boolean models, we will only briefly recall the original encoding here.
Its basis is that every \emph{gene} of the original model is represented by two separate places, corresponding to its two states, active, and inactive.
Each conjunct of the logical function that activate the \emph{gene} will lead to a transition, consuming the inactive place, producing the active place, and with all other literals both consumed and produced.
And conversely for the inactivation.

Note that given a Logical model in the standard SBML-Qual format~\cite{chaouiya2013sbml}, i.e., one of the packages of SBML v3~\cite{keating2020sbml}, one can easily obtain its Petri-net encoding in the PNML\footnote{\url{https://www.pnml.org/}} standard using the BioLQM\footnote{\url{http://www.colomoto.org/biolqm/}} library.
This piece of software extracted from GINsim~\cite{chaouiya2012logical} and part of the CoLoMoTo\footnote{\url{http://colomoto.org/}} software suite allows for easy conversion between standard formats.
It also accepts many other common formats for Logical models, notably the \verb|.bnet| files of the  BoolNet~\cite{mussel2010boolnet,klarner2017pyboolnet} tools.

Conversion is executed as follows:

\noindent{\small \verb|java -jar GINsim.jar -lqm <input.{sbml,bnet,zginml,...}> <output.pnml>|}

\subsubsection{Siphons}

Siphons are a classical property of place-transition nets~\cite{peterson81petri}.
Note however that the use of siphons for the analysis of biological models, though it is not new~\cite{angeli2007petri,angeli2011persistence,degrand2020graphical}, has been mostly relevant to the ODE-based continuous semantics of Chemical Reaction Networks.

We recall here the basic definition:

\begin{definition}

  A siphon is a set of places such that for each edge from a transition to any place of the siphon, there is an edge from a place of the siphon to that transition.

\end{definition}

Intuitively a siphon is a set of places that once empty remains empty, i.e.,
a set of species that cannot be produced again once they have been completely
consumed.

\subsection{New method}


\section{Evaluation}
\subsection{Benchmark vs. implicant-based methods}

Using BioLQM~\ref{sec:encoding}

\url{http://colomoto.org/biolqm/doc/tools-trapspace.html}

\subsection{Detailed biological example}

TBD

\section{Conclusion}

\subsubsection{Acknowledgments}
Coffee is good, you should drink some!

\bibliographystyle{splncs04}
\bibliography{cmsb22.bib}

\end{document}

% \begin{table}
% \caption{Table captions should be placed above the
% tables.}\label{tab1}
% \begin{tabular}{|l|l|l|}
% \hline
% Heading level &  Example & Font size and style\\
% \hline
% Title (centered) &  {\Large\bfseries Lecture Notes} & 14 point, bold\\
% 1st-level heading &  {\large\bfseries 1 Introduction} & 12 point, bold\\
% 2nd-level heading & {\bfseries 2.1 Printing Area} & 10 point, bold\\
% 3rd-level heading & {\bfseries Run-in Heading in Bold.} Text follows & 10 point, bold\\
% 4th-level heading & {\itshape Lowest Level Heading.} Text follows & 10 point, italic\\
% \hline
% \end{tabular}
% \end{table}


% \noindent Displayed equations are centered and set on a separate
% line.
% \begin{equation}
% x + y = z
% \end{equation}
% Please try to avoid rasterized images for line-art diagrams and
% schemas. Whenever possible, use vector graphics instead (see
% Fig.~\ref{fig1}).

% \begin{figure}
% % \includegraphics[width=\textwidth]{fig1.eps}
% \caption{A figure caption is always placed below the illustration.
% Please note that short captions are centered, while long ones are
% justified by the macro package automatically.} \label{fig1}
% \end{figure}

% \begin{theorem}
% This is a sample theorem. The run-in heading is set in bold, while
% the following text appears in italics. Definitions, lemmas,
% propositions, and corollaries are styled the same way.
% \end{theorem}
% %
% % the environments 'definition', 'lemma', 'proposition', 'corollary',
% % 'remark', and 'example' are defined in the LLNCS documentclass as well.
% %
% \begin{proof}
% Proofs, examples, and remarks have the initial word in italics,
% while the following text appears in normal font.
% \end{proof}
